Prema \citep{dietterich1998approximate} postoji devet temeljnih statističkih pitanja u strojnom učenju, a prikazana su grafički na slici \ref{fig:stat_questions}. Pitanja su hijerarhijski organizirana u četiri razine, razina: 
\begin{itemize}
\item domene -- jednolika ili raznolika,
\item tehnike -- odabir klasifikatora ili algoritma učenja,
\item vrste zadatka -- predviđanje ili usporedba algoritama/klasifikatora i 
\item veličine populacije -- mala ili velika.
 . 
\end{itemize}
Svaki list stabla predstavlja jedno pitanje, a do pitanja se dolazi uzimanjem u obzir svih prethodnika lista. Primjerice, kako predvidjeti preciznost klasifikatora u velikom, jednolikom skupu podataka je pitanje označeno brojem $1$ na slici. Također, bitno je napomenuti kako je bitan faktor i brojnost korištenih metoda tj. broj algoritama učenja ili klasifikatora. %TODO dopisati


\section{Terminologija}

U ovom odjeljku objasnit ćemo pojmove vezane za strojno učenje korištene u ostatku rada. Postoji mnogo referenci za upoznavanje s područjem strojnog učenja, kao što su \citep{anderson1986machine} i \citep{bishop2006pattern}. Kako se strojno učenje primjenjuje u sve više domena, tako postoji i literatura za upoznavanje sa strojnim učenjem, kao što je \citep{baldi2001bioinformatics}. 

Razlika između algoritma učenja i klasifikatora je u tome što je klasifikator istrenirani proizvod bez mogućnosti daljnjeg razvitka. Algoritam učenja može proizvesti jedan ili više klasfikatora. Ponekad se uspoređuju algoritmi učenja, a ponekad klasifikatori, ovisno o primjeni. Primjerice, potrebno je izgraditi stroj za automatsko prepoznavanje vrsta riječi u obradi prirodnog teksta: moguće je pokušati uporabiti različite algoritme učenja za dobivanje klasifikatora. Moguće je uspoređivati rezultate algoritama učenja, ali i samih klasifikatora. U kontekstu učenja kod čovjeka, moguće je uspoređivati tehnike učenja, primjerice tzv. kampanjski način učenja s redovitim učenjem, što bi odgovaralo usporedbi algoritama učenja. Uspoređivanje studenata i njihovih rezultata na ispitima odgovara usporedbi klasifikatora.

Moguće je uspoređivati najmanje dva algoritma učenja ili klasifikatora, ali i više. Prilikom usporedbe dvaju algoritama učenja koristit će se imena $A$ i $B$, a proizvodi tih algoritama učenja, klasifikatori, imat će oznate $C_A$ i $C_B$. Niz cjelobrojnih vrijednosti $1\dots i$ služiti razlikovanju više od dva algoritma učenja, a pripadajući klasifikatori označavat će se sa $C_1 \dots C_i$. Skup podataka bit će označen znakom $S$, a njegova veličina oznakom $n$. Svi ostali pojmovi bit će objašnjeni u odjeljku u kojem se prvi put spominju.

\begin{figure}[h]
\centering
% Graphic for TeX using PGF
% Title: /home/filip/Documents/Doktorski/seminari/statističko testiranje/Diagram1.dia
% Creator: Dia v0.97.2
% CreationDate: Mon Jan 13 23:44:12 2014
% For: filip
% \usepackage{tikz}
% The following commands are not supported in PSTricks at present
% We define them conditionally, so when they are implemented,
% this pgf file will use them.
\ifx\du\undefined
  \newlength{\du}
\fi
\setlength{\du}{15\unitlength}
\begin{tikzpicture}
\pgftransformxscale{1.000000}
\pgftransformyscale{-1.000000}
\definecolor{dialinecolor}{rgb}{0.000000, 0.000000, 0.000000}
\pgfsetstrokecolor{dialinecolor}
\definecolor{dialinecolor}{rgb}{1.000000, 1.000000, 1.000000}
\pgfsetfillcolor{dialinecolor}
% setfont left to latex
\definecolor{dialinecolor}{rgb}{0.000000, 0.000000, 0.000000}
\pgfsetstrokecolor{dialinecolor}
\node at (19.200000\du,5.812500\du){jednolika};
% setfont left to latex
\definecolor{dialinecolor}{rgb}{0.000000, 0.000000, 0.000000}
\pgfsetstrokecolor{dialinecolor}
\node at (19.200000\du,6.612500\du){domena};
% setfont left to latex
\definecolor{dialinecolor}{rgb}{0.000000, 0.000000, 0.000000}
\pgfsetstrokecolor{dialinecolor}
\node at (25.150000\du,5.862500\du){raznovrsna };
% setfont left to latex
\definecolor{dialinecolor}{rgb}{0.000000, 0.000000, 0.000000}
\pgfsetstrokecolor{dialinecolor}
\node at (25.150000\du,6.662500\du){domena};
% setfont left to latex
\definecolor{dialinecolor}{rgb}{0.000000, 0.000000, 0.000000}
\pgfsetstrokecolor{dialinecolor}
\node at (25.150000\du,7.462500\du){             9            };
% setfont left to latex
\definecolor{dialinecolor}{rgb}{0.000000, 0.000000, 0.000000}
\pgfsetstrokecolor{dialinecolor}
\node at (12.800000\du,21.150000\du){mali};
% setfont left to latex
\definecolor{dialinecolor}{rgb}{0.000000, 0.000000, 0.000000}
\pgfsetstrokecolor{dialinecolor}
\node at (12.800000\du,21.950000\du){uzorak};
% setfont left to latex
\definecolor{dialinecolor}{rgb}{0.000000, 0.000000, 0.000000}
\pgfsetstrokecolor{dialinecolor}
\node at (12.800000\du,22.750000\du){        2        };
% setfont left to latex
\definecolor{dialinecolor}{rgb}{0.000000, 0.000000, 0.000000}
\pgfsetstrokecolor{dialinecolor}
\node at (9.350000\du,21.200000\du){velik};
% setfont left to latex
\definecolor{dialinecolor}{rgb}{0.000000, 0.000000, 0.000000}
\pgfsetstrokecolor{dialinecolor}
\node at (9.350000\du,22.000000\du){uzorak};
% setfont left to latex
\definecolor{dialinecolor}{rgb}{0.000000, 0.000000, 0.000000}
\pgfsetstrokecolor{dialinecolor}
\node at (9.350000\du,22.800000\du){1};
% setfont left to latex
\definecolor{dialinecolor}{rgb}{0.000000, 0.000000, 0.000000}
\pgfsetstrokecolor{dialinecolor}
\node at (18.750000\du,21.250000\du){mali};
% setfont left to latex
\definecolor{dialinecolor}{rgb}{0.000000, 0.000000, 0.000000}
\pgfsetstrokecolor{dialinecolor}
\node at (18.750000\du,22.050000\du){uzorak};
% setfont left to latex
\definecolor{dialinecolor}{rgb}{0.000000, 0.000000, 0.000000}
\pgfsetstrokecolor{dialinecolor}
\node at (18.750000\du,22.850000\du){4};
% setfont left to latex
\definecolor{dialinecolor}{rgb}{0.000000, 0.000000, 0.000000}
\pgfsetstrokecolor{dialinecolor}
\node at (15.450000\du,21.250000\du){velik};
% setfont left to latex
\definecolor{dialinecolor}{rgb}{0.000000, 0.000000, 0.000000}
\pgfsetstrokecolor{dialinecolor}
\node at (15.450000\du,22.050000\du){uzorak};
% setfont left to latex
\definecolor{dialinecolor}{rgb}{0.000000, 0.000000, 0.000000}
\pgfsetstrokecolor{dialinecolor}
\node at (15.450000\du,22.850000\du){3};
% setfont left to latex
\definecolor{dialinecolor}{rgb}{0.000000, 0.000000, 0.000000}
\pgfsetstrokecolor{dialinecolor}
\node at (21.900000\du,21.200000\du){velik};
% setfont left to latex
\definecolor{dialinecolor}{rgb}{0.000000, 0.000000, 0.000000}
\pgfsetstrokecolor{dialinecolor}
\node at (21.900000\du,22.000000\du){uzorak};
% setfont left to latex
\definecolor{dialinecolor}{rgb}{0.000000, 0.000000, 0.000000}
\pgfsetstrokecolor{dialinecolor}
\node at (21.900000\du,22.800000\du){5};
% setfont left to latex
\definecolor{dialinecolor}{rgb}{0.000000, 0.000000, 0.000000}
\pgfsetstrokecolor{dialinecolor}
\node at (25.200000\du,21.250000\du){mali};
% setfont left to latex
\definecolor{dialinecolor}{rgb}{0.000000, 0.000000, 0.000000}
\pgfsetstrokecolor{dialinecolor}
\node at (25.200000\du,22.050000\du){uzorak};
% setfont left to latex
\definecolor{dialinecolor}{rgb}{0.000000, 0.000000, 0.000000}
\pgfsetstrokecolor{dialinecolor}
\node at (25.200000\du,22.850000\du){6};
% setfont left to latex
\definecolor{dialinecolor}{rgb}{0.000000, 0.000000, 0.000000}
\pgfsetstrokecolor{dialinecolor}
\node at (28.400000\du,21.250000\du){velik};
% setfont left to latex
\definecolor{dialinecolor}{rgb}{0.000000, 0.000000, 0.000000}
\pgfsetstrokecolor{dialinecolor}
\node at (28.400000\du,22.050000\du){uzorak};
% setfont left to latex
\definecolor{dialinecolor}{rgb}{0.000000, 0.000000, 0.000000}
\pgfsetstrokecolor{dialinecolor}
\node at (28.400000\du,22.850000\du){7};
% setfont left to latex
\definecolor{dialinecolor}{rgb}{0.000000, 0.000000, 0.000000}
\pgfsetstrokecolor{dialinecolor}
\node at (31.000000\du,21.250000\du){mali};
% setfont left to latex
\definecolor{dialinecolor}{rgb}{0.000000, 0.000000, 0.000000}
\pgfsetstrokecolor{dialinecolor}
\node at (31.000000\du,22.050000\du){uzorak};
% setfont left to latex
\definecolor{dialinecolor}{rgb}{0.000000, 0.000000, 0.000000}
\pgfsetstrokecolor{dialinecolor}
\node at (31.000000\du,22.850000\du){8};
% setfont left to latex
\definecolor{dialinecolor}{rgb}{0.000000, 0.000000, 0.000000}
\pgfsetstrokecolor{dialinecolor}
\node at (11.900000\du,15.450000\du){predviđanje};
% setfont left to latex
\definecolor{dialinecolor}{rgb}{0.000000, 0.000000, 0.000000}
\pgfsetstrokecolor{dialinecolor}
\node at (11.900000\du,16.250000\du){preciznosti };
% setfont left to latex
\definecolor{dialinecolor}{rgb}{0.000000, 0.000000, 0.000000}
\pgfsetstrokecolor{dialinecolor}
\node at (11.900000\du,17.050000\du){klasifikatora};
% setfont left to latex
\definecolor{dialinecolor}{rgb}{0.000000, 0.000000, 0.000000}
\pgfsetstrokecolor{dialinecolor}
\node at (17.000000\du,15.450000\du){izbor};
% setfont left to latex
\definecolor{dialinecolor}{rgb}{0.000000, 0.000000, 0.000000}
\pgfsetstrokecolor{dialinecolor}
\node at (17.000000\du,16.250000\du){klasifikatora};
% setfont left to latex
\definecolor{dialinecolor}{rgb}{0.000000, 0.000000, 0.000000}
\pgfsetstrokecolor{dialinecolor}
\node at (17.000000\du,17.050000\du){};
% setfont left to latex
\definecolor{dialinecolor}{rgb}{0.000000, 0.000000, 0.000000}
\pgfsetstrokecolor{dialinecolor}
\node at (22.800000\du,15.500000\du){predviđanje};
% setfont left to latex
\definecolor{dialinecolor}{rgb}{0.000000, 0.000000, 0.000000}
\pgfsetstrokecolor{dialinecolor}
\node at (22.800000\du,16.300000\du){preciznosti};
% setfont left to latex
\definecolor{dialinecolor}{rgb}{0.000000, 0.000000, 0.000000}
\pgfsetstrokecolor{dialinecolor}
\node at (22.800000\du,17.100000\du){algoritma};
% setfont left to latex
\definecolor{dialinecolor}{rgb}{0.000000, 0.000000, 0.000000}
\pgfsetstrokecolor{dialinecolor}
\node at (28.550000\du,16.000000\du){izbor algoritma};
% setfont left to latex
\definecolor{dialinecolor}{rgb}{0.000000, 0.000000, 0.000000}
\pgfsetstrokecolor{dialinecolor}
\node at (23.150000\du,11.050000\du){analiza algoritama};
% setfont left to latex
\definecolor{dialinecolor}{rgb}{0.000000, 0.000000, 0.000000}
\pgfsetstrokecolor{dialinecolor}
\node at (15.150000\du,11.100000\du){analiza klasifikatora};
\pgfsetlinewidth{0.100000\du}
\pgfsetdash{}{0pt}
\pgfsetdash{}{0pt}
\pgfsetbuttcap
{
\definecolor{dialinecolor}{rgb}{0.000000, 0.000000, 0.000000}
\pgfsetfillcolor{dialinecolor}
% was here!!!
\definecolor{dialinecolor}{rgb}{0.000000, 0.000000, 0.000000}
\pgfsetstrokecolor{dialinecolor}
\draw (22.300000\du,2.062500\du)--(20.350000\du,4.962500\du);
}
\pgfsetlinewidth{0.100000\du}
\pgfsetdash{}{0pt}
\pgfsetdash{}{0pt}
\pgfsetbuttcap
{
\definecolor{dialinecolor}{rgb}{0.000000, 0.000000, 0.000000}
\pgfsetfillcolor{dialinecolor}
% was here!!!
\definecolor{dialinecolor}{rgb}{0.000000, 0.000000, 0.000000}
\pgfsetstrokecolor{dialinecolor}
\draw (22.100000\du,2.062500\du)--(24.200000\du,5.012500\du);
}
\pgfsetlinewidth{0.100000\du}
\pgfsetdash{}{0pt}
\pgfsetdash{}{0pt}
\pgfsetbuttcap
{
\definecolor{dialinecolor}{rgb}{0.000000, 0.000000, 0.000000}
\pgfsetfillcolor{dialinecolor}
% was here!!!
\definecolor{dialinecolor}{rgb}{0.000000, 0.000000, 0.000000}
\pgfsetstrokecolor{dialinecolor}
\draw (14.749400\du,11.392200\du)--(12.799400\du,14.292200\du);
}
\pgfsetlinewidth{0.100000\du}
\pgfsetdash{}{0pt}
\pgfsetdash{}{0pt}
\pgfsetbuttcap
{
\definecolor{dialinecolor}{rgb}{0.000000, 0.000000, 0.000000}
\pgfsetfillcolor{dialinecolor}
% was here!!!
\definecolor{dialinecolor}{rgb}{0.000000, 0.000000, 0.000000}
\pgfsetstrokecolor{dialinecolor}
\draw (14.549400\du,11.392200\du)--(16.649400\du,14.342200\du);
}
\pgfsetlinewidth{0.100000\du}
\pgfsetdash{}{0pt}
\pgfsetdash{}{0pt}
\pgfsetbuttcap
{
\definecolor{dialinecolor}{rgb}{0.000000, 0.000000, 0.000000}
\pgfsetfillcolor{dialinecolor}
% was here!!!
\definecolor{dialinecolor}{rgb}{0.000000, 0.000000, 0.000000}
\pgfsetstrokecolor{dialinecolor}
\draw (11.189400\du,17.347200\du)--(9.239390\du,20.247200\du);
}
\pgfsetlinewidth{0.100000\du}
\pgfsetdash{}{0pt}
\pgfsetdash{}{0pt}
\pgfsetbuttcap
{
\definecolor{dialinecolor}{rgb}{0.000000, 0.000000, 0.000000}
\pgfsetfillcolor{dialinecolor}
% was here!!!
\definecolor{dialinecolor}{rgb}{0.000000, 0.000000, 0.000000}
\pgfsetstrokecolor{dialinecolor}
\draw (10.989400\du,17.347200\du)--(13.089400\du,20.297200\du);
}
\pgfsetlinewidth{0.100000\du}
\pgfsetdash{}{0pt}
\pgfsetdash{}{0pt}
\pgfsetbuttcap
{
\definecolor{dialinecolor}{rgb}{0.000000, 0.000000, 0.000000}
\pgfsetfillcolor{dialinecolor}
% was here!!!
\definecolor{dialinecolor}{rgb}{0.000000, 0.000000, 0.000000}
\pgfsetstrokecolor{dialinecolor}
\draw (16.979400\du,17.302200\du)--(15.029400\du,20.202200\du);
}
\pgfsetlinewidth{0.100000\du}
\pgfsetdash{}{0pt}
\pgfsetdash{}{0pt}
\pgfsetbuttcap
{
\definecolor{dialinecolor}{rgb}{0.000000, 0.000000, 0.000000}
\pgfsetfillcolor{dialinecolor}
% was here!!!
\definecolor{dialinecolor}{rgb}{0.000000, 0.000000, 0.000000}
\pgfsetstrokecolor{dialinecolor}
\draw (16.779400\du,17.302200\du)--(18.879400\du,20.252200\du);
}
\pgfsetlinewidth{0.100000\du}
\pgfsetdash{}{0pt}
\pgfsetdash{}{0pt}
\pgfsetbuttcap
{
\definecolor{dialinecolor}{rgb}{0.000000, 0.000000, 0.000000}
\pgfsetfillcolor{dialinecolor}
% was here!!!
\definecolor{dialinecolor}{rgb}{0.000000, 0.000000, 0.000000}
\pgfsetstrokecolor{dialinecolor}
\draw (24.919400\du,11.557200\du)--(22.969400\du,14.457200\du);
}
\pgfsetlinewidth{0.100000\du}
\pgfsetdash{}{0pt}
\pgfsetdash{}{0pt}
\pgfsetbuttcap
{
\definecolor{dialinecolor}{rgb}{0.000000, 0.000000, 0.000000}
\pgfsetfillcolor{dialinecolor}
% was here!!!
\definecolor{dialinecolor}{rgb}{0.000000, 0.000000, 0.000000}
\pgfsetstrokecolor{dialinecolor}
\draw (24.719400\du,11.557200\du)--(26.819400\du,14.507200\du);
}
\pgfsetlinewidth{0.100000\du}
\pgfsetdash{}{0pt}
\pgfsetdash{}{0pt}
\pgfsetbuttcap
{
\definecolor{dialinecolor}{rgb}{0.000000, 0.000000, 0.000000}
\pgfsetfillcolor{dialinecolor}
% was here!!!
\definecolor{dialinecolor}{rgb}{0.000000, 0.000000, 0.000000}
\pgfsetstrokecolor{dialinecolor}
\draw (23.059400\du,17.362200\du)--(21.109400\du,20.262200\du);
}
\pgfsetlinewidth{0.100000\du}
\pgfsetdash{}{0pt}
\pgfsetdash{}{0pt}
\pgfsetbuttcap
{
\definecolor{dialinecolor}{rgb}{0.000000, 0.000000, 0.000000}
\pgfsetfillcolor{dialinecolor}
% was here!!!
\definecolor{dialinecolor}{rgb}{0.000000, 0.000000, 0.000000}
\pgfsetstrokecolor{dialinecolor}
\draw (22.859400\du,17.362200\du)--(24.959400\du,20.312200\du);
}
\pgfsetlinewidth{0.100000\du}
\pgfsetdash{}{0pt}
\pgfsetdash{}{0pt}
\pgfsetbuttcap
{
\definecolor{dialinecolor}{rgb}{0.000000, 0.000000, 0.000000}
\pgfsetfillcolor{dialinecolor}
% was here!!!
\definecolor{dialinecolor}{rgb}{0.000000, 0.000000, 0.000000}
\pgfsetstrokecolor{dialinecolor}
\draw (29.699400\du,17.367200\du)--(27.749400\du,20.267200\du);
}
\pgfsetlinewidth{0.100000\du}
\pgfsetdash{}{0pt}
\pgfsetdash{}{0pt}
\pgfsetbuttcap
{
\definecolor{dialinecolor}{rgb}{0.000000, 0.000000, 0.000000}
\pgfsetfillcolor{dialinecolor}
% was here!!!
\definecolor{dialinecolor}{rgb}{0.000000, 0.000000, 0.000000}
\pgfsetstrokecolor{dialinecolor}
\draw (29.449400\du,17.367200\du)--(31.549400\du,20.317200\du);
}
\pgfsetlinewidth{0.100000\du}
\pgfsetdash{}{0pt}
\pgfsetdash{}{0pt}
\pgfsetbuttcap
{
\definecolor{dialinecolor}{rgb}{0.000000, 0.000000, 0.000000}
\pgfsetfillcolor{dialinecolor}
% was here!!!
\definecolor{dialinecolor}{rgb}{0.000000, 0.000000, 0.000000}
\pgfsetstrokecolor{dialinecolor}
\draw (19.200000\du,7.000000\du)--(17.250000\du,9.900000\du);
}
\pgfsetlinewidth{0.100000\du}
\pgfsetdash{}{0pt}
\pgfsetdash{}{0pt}
\pgfsetbuttcap
{
\definecolor{dialinecolor}{rgb}{0.000000, 0.000000, 0.000000}
\pgfsetfillcolor{dialinecolor}
% was here!!!
\definecolor{dialinecolor}{rgb}{0.000000, 0.000000, 0.000000}
\pgfsetstrokecolor{dialinecolor}
\draw (19.000000\du,7.000000\du)--(21.100000\du,9.950000\du);
}
\end{tikzpicture}

\caption{Statistička pitanja u strojnom učenju}
\label{fig:stat_questions}
\end{figure}

\section{Statistička pitanja usporedbe u strojnom učenju}

Ovaj odjeljak pokušat će odgovoriti na statistička pitanja vezana uz usporedbu algoritama učenja/klasifikatora u strojnom učenju ($3$, $4$, $7$, $8$), usporediti algoritme učenja i klasifikatore nad raznolikim skupovima podataka (pitanje $9$) te opisati načine ispitivanja statističke značajnosti u okruženju s više od 2 algoritma učenja ili klasifikatora.

\subsection{Pitanje 3 -- usporedba klasifikatora na velikom skupu podataka}
\label{subsec:pitanje3}

Konstruirani su klasifikatori $C_A$ i $C_B$. Moguće je izdvojiti zaseban skup podataka $T$ nad kojim će se testirati dobiveni klasifikatori. Potrebno je mjeriti performanse svakog klasifikatora na posebnom skupu za testiranje te primjeniti McNemarov test.

\subsubsection{McNemarov test}

Za McNemarov test, opisan u \citep{everitt1992analysis} potrebno je populaciju $S$ podijeliti na skupove za treniranje $R$ i testiranj $T$. Algoritmi $A$ i $B$ proizvode klasifikatore $C_A$ i $C_B$ koji produciraju rezultate klasifikacije nad skupom podataka za testiranje. Provođenje testa zahtjeva izgradnju tablice kontigencije:

\begin{table}[h]
\centering
\begin{tabular}{{| >{\centering\arraybackslash}m{6cm} | >{\centering\arraybackslash}m{6cm} |}}
\hline
broj primjera koje su $C_A$ i $C_B$ pogrešno klasificirali & broj primjera koje je $C_A$ pogrešno, a $C_B$ ispravno klasificirao \\ \hline
broj primjera koje je $C_B$ pogrešno, a $C_A$ ispravno klasificirao & broj primjera koje su $C_A$ i $C_B$ ispravno klasificirali \\
\hline
\end{tabular}
\end{table}
skraćeno:
\begin{table}[h]
\centering
\begin{tabular}{|c|c|}
\hline
$n_{00}$ & $n_{01}$ \\ \hline
$n_{10}$ & $n_{11}$ \\
\hline
\end{tabular}
\end{table}
gdje je $n = n_{00}+n_{01}+n_{10}+n_{11}$ ukupan broj primjera u skupu za testiranje $T$. Prema nultoj hipotezi, oba klasifikatora trebala bi imati jednak broj pogrešaka $n_{10}=n_{01}$. McNemarov test, zasnovan na $\chi^2$ testu, ispituje koliko vjerojatnost raspodjele broja pogrešno klasificiranih primjera prema nultoj hipotezi, a očekivana distribucija može se prikazati kontigencijskom tablicom:
\begin{table}[h]
\centering
\begin{tabular}{|c|c|}
\hline
$n_{00}$ & $(n_{01} + n_{10})/2$ \\ \hline
$(n_{01} + n_{10})/2$ & $n_{11}$ \\
\hline
\end{tabular}
\end{table}

U ovom slučaju $\chi^2$ statistika iznosi:
\begin{equation}
\frac{(|n_{01} - n_{10}|-1)^2}{n_{01}+n_{10}}
\end{equation}

Kako bi se odbacila nulta hipoteza, potrebno je da vrijednost dobivena iz testnog skupa podataka veća od granične vrijednosti odgovarajućih stupnjeva slobode ($\chi^{2}_{1,0.95} = 3.84$).

\subsection{Pitanje 4 -- usporedba klasifikatora na malom skupu podataka}

Dva klasifikatora, $C_A$ i $C_B$ dobiveni su treniranjem algoritama učenja $A$ i $B$ nad malim skupom podataka $S$. Ovaj problem \citep{dietterich1998approximate} izjednačava s pitanjem 8 u odjeljku \ref{subsec:pitanje8}, gdje će biti objašnjene potencijalne metode za rješenje problema. 

\subsection{Pitanje 7 -- usporedba algoritama učenja na velikom skupu podataka}

Algoritmi učenja $A$ i $B$ treniraju se i testiraju na velikom skupu podataka $S$. Odgovor na ovo pitanje dobiveno je u istraživanju u okviru DELVE projekta \citep{hintondelve}: potrebno je podijeliti skup $S$ na disjunktne skupove za trening te izdvojiti jedan skup za test. Klasifikatori dobiveni treniranjem algoritama učenja nad svim skupovima za treniranje se ispituju skupom za testiranje. Moguće je primijeniti analizu varijance s varijablama \begin{itemize}
\item izbora algoritma učenja,
\item izbora skupa za treniranje i 
\item za svaku jedinku iz skupa za testiranje.
\end{itemize}
Kvazi-$F$ testom \citep{eisen1966225} se utvrđuje postoji li statistička značajna razlika između različitih algoritama učenja.

\subsection{Pitanje 8 -- usporedba algoritama učenja na malom skupu podataka}
\label{subsec:pitanje8}

Dva algoritma učenja $A$ i $B$ potrebno je istrenirati na skupu $S$ i provjeriti koji će od algoritama davati bolje rezultate na novom skupu jednake veličine kao i $S$. U radu \citep{dietterich1998approximate} opisuje se pet različitih statističkih testova za usporedbu algoritama učenja s ograničenim skupom podataka. 

\subsubsection{McNemarov test}

Ovdje je ponovno moguće je iskoristiti McNemarov test opisan u odjeljku \ref{subsec:pitanje3}. No, nedostatak McNemarovog testa je činjenica da se skup za treniranje $R$ ($|R| << |S|)$ odabire jednom, a nad njim se potom rade sve usporedbe. Prema tome, McNemarov test je primjenjiv ukoliko je varijabilnost podataka mala, jer se radi o podskupu mnogo većeg (populacijskog) skupa $S$. Pretpostavlja se kako sve što vrijedi za odabrani skup $R$, vrijedi i za čitavu populaciju $S$, što ne mora biti istina u praksi. 

\subsubsection{Test razlika}

Test razlika \engl{Test for the difference of two proportions} \citep{snedecor1980statistical} mjeri razinu pogreške kod algoritama $A$ i $B$ preko vjerojatnosti:
\begin{equation}
p_A = \frac{n_{00} + n_{01}}{n}
\end{equation}
\begin{equation}
p_B = \frac{n_{00} + n_{10}}{n}
\end{equation}

Ako su $p_A$ i $p_B$ su vjerojatnosti pogrešne klasifikacije algoritama $A$ i $B$, broj pogrešno klasificiranih $N$ primjera zadovoljava binomnu raspodjelu srednje vrijednosti $Np_A$ i varijance $p_A(1-p_A)N$. Binomna raspodjela prelazi u normalnu za dovoljno velik broj primjera $n$. Ako pretpostavimo da su $p_A$ i $p_B$ nezavisni, veličina $p=(p_A+p_B)/2$ također zadovoljava normalnu raspodjelu. Da bi nulta hipoteza bila zadovoljena, srednja vrijednost izvedene veličine $p$ bit će nula, a standardna pogreška i $z$ vrijednost:
\begin{equation}
se = \sqrt{\frac{2p(1-p)}{n}}.
\end{equation}
\begin{equation}
z = \frac{p_A-p_B}{\sqrt{2p(1-p)/n}}.
\end{equation}
Nulta hipoteza se odbija je dobivena $z$ vrijednost van graničnih vrijednosti ($|z| > Z_{0.975} = 1.96$ kod dvostranog testa s $\alpha=0.05$).

Prilikom postavljanja testa, pogrešno se pretpostavlja nezavisnost mjera $p_A$ i $p_B$, mjerenih nad istim skupom podataka $T$. Također, test, kao i McNemarov, uzima u obzir jedinstveni skup za treniranje $R$.	

\subsubsection{T-test s resempliranjem}

T-test s resempliranjem je najkorišteniji test u strojnom učenju, prema \citep{dietterich1998approximate}. Prvi korak je izrada uzoraka $R$, unaprijed poznate veličine, iz skupa $S$ i kreiranje testnih skupova $T$ iz dobivenih uzoraka. Uzorkovanje se ponavlja; često korišten broj ponavljanja je 30. $p_{A}^{i}$ i $p_{B}^{i}$ su vjerojatnosti pogrešne klasifikacije algoritama klasifikatora dobivenih algoritmima $A$ i $B$ nad $i$-tim skupom podataka. Pod pretpostavkom da je uzorkovanje neovisno mjera $p^{(i)} = p^{(i)}_{A} - p_{B}^{(i)}$ moguće je primjeniti Studentov t-test, gdje se t-statistika dobije formulom:
\begin{equation}
\hat{p} = \frac{1}{n}\sum_{i=1}^{n} p^{(i)}
\end{equation}
\begin{equation}
t = \frac{\hat{p} \cdot \sqrt{n}}{\sum_{i=1}^{n} (p^{(i)}-\hat{p})^2}.
\end{equation}
Nulta hipoteza se odbija ako je dobivena $t$ vrijednost veća od odgovarajuće (tablične) $t$ vrijednosti (primjerice, $|t|>t_{29,0.975} = 2.05$ za 30 uzorkovanja). Ponovno, radi se pretpostavka nezavisnosti varijabli $p_{A}^{(i)}$ i $p_{B}^{(i)}$ do kojih se dolazi uzorkovanjem iz istog skupa podataka. Druga pogrešna pretpostavka nezavisnosti radi se prilikom zbrajanja $p^{(i)}$ vrijednosti iz različitih uzorkovanja. Ovdje pretpostavka nezavisnosti ne stoji zbog mogućih preklapanja uzoraka.

\subsubsection{Upareni t-test s $k$-strukom unakrsnom validacijom}

Jedna od mana t-testa s resempliranjem je (ponekad) neispravna pretpostavka o nezavisnosti podataka u različitim uzorcima. Upareni t-test s $k$-strukom unakrsnom validacijom dijeli skup podataka $S$ na $k$ disjunktnih skupova za testiranje jednake veličine $T_1, \dots ,T_k$. Skup za treniranje u koraku $i$ dobiva se unijom svih skupova $T_j, j\neq i$. Sve ostalo identično je kao i u prethodno objašnjenom t-testu. Prema \citep{dietterich1998approximate} jedina mana ovog testa je visoki korelacijski faktor između različitih skupova za treniranje. 

\subsubsection{5x2cv upareni t-test}

Kako bi prebrodili mane uparenog t-testa s $k$-strukom unakrsnom validacijom \citep{dietterich1998approximate} je primijetio kako bilo kakvo preklapanje u skupovima podataka za testiranje ili treniranje narušava performanse testa. Iz tog razloga, osmišljen je 5x2cv upareni t-test. U tom testu pet puta se ponavlja unakrsna validacija s dva preklapanja \engl{5 replications of 2-fold cross validation}. Prilikom svake podjele skupa podataka $S$ dobivaju se skupovi $S_1$ i $S_2$. Algoritmi $A$ i $B$ koriste oba skupa koriste se za treniranje i testiranje. Ispitivanjem dobivenih klasifikatora dobiju se vjerojatnosti: $p_{A}^{(1)}$, $p_{A}^{(2)}$, $p_{B}^{(1)}$, $p_{B}^{(2)}$, iz njih razlike $p^{(1)}=p_{A}^{(1)}-p_{B}^{(1)}$ i $p^{(2)}=p_{A}^{(2)}-p_{B}^{(2)}$, iz razlika varijanca $s^2=(p^{(1)}-\hat{p})^2+(p^{(2)} - \hat{p})^2$, s time da je $\hat{p} = (p^{(1)}+p^{(2)})/2$. $s_{i}^{2}$ je varijanca dobivena u $i$-itom (od 5) ponavljanju. Iz toga moguće je dobiti vrijednost t statistike, 5x2cv $\tilde{t}$ vrijednost iz:
\begin{equation}
\tilde{t} = \frac{p_{1}^{(1)}}{\sqrt{\frac{1}{5}\sum_{i=1}^{5}s_{i}^{2}}}
\end{equation}
Prema nultoj hipotezi dobivena 5x2cv $\tilde{t}$ statistika poprima obilježja t-raspodjele s pet stupnjeva slobode. \citep{dietterich1998approximate} je teorijski i eksperimentalno dokazao ovu tvrdnju. Alpaydm u svom radu \citep{alpaydm1999combined}  primjećuje kako vrijednost dobivene $\tilde{t}$ statistike ovisi o izboru $p_{i}^{(j)}$ vjerojatnosti (ovdje $p_{1}^{(1)}$. Iz tog razloga predlaže poboljšanje 5x2cv t-testa, kombinirani 5x2cv $F$ test koji uzima u obzir sve dobivene $p_{i}^{(j)}$ vjerojatnosti. 

\subsection{Pitanje 9 -- usporedba na raznolikom skupu podataka}
\label{subsec:couting}

Analizom objavljenih radova na konferencijama \textit{International Conference of Machine Learning} između 1999. i 2003. godine \citep{demvsar2006statistical} detektirane su metode usporedbi klasifikatora nad različitim skupovima podataka:
\begin{itemize}
\item prosjek preciznosti u različitim skupovima podataka,
\item t-test,
\item upareni t-test jedan protiv ostalih,
\item upareni t-test svatko protiv svakog,
\item prebrojavanje pobjeda/nerješenih rezultata/poraza i
\item prebrojavanje statistički značajnih pobjeda/nerješenih rezultata/poraza
\end{itemize}

Opće je prihvaćeno da je korištenje t-testa za usporedbu klasifikatora na različitim skupovima podataka neprimjereno. U radu \citep{demvsar2006statistical} preporučuje se Wilcoxonov test rangova s predznacima (engl.\,\textit{Wilcoxon ranked sign test}) \citep{wilcoxon1945individual}. Uz njega, koristi se i test s predznacima \cite{dixon1946statistical}. 

U idućim pododjeljcima opisat će se različiti načini mjerenja statističke značajnosti viđeni u znanstvenim publikacijama.

\subsubsection{Računanje prosjeka}

Mjerenje performansi klasifikatora nad različitim skupovima podataka moguće je jednostavno zbrojiti i uzeti srednju vrijednost:
\begin{equation}
\hat{X} = \frac{1}{N} \sum^{N}_{i = 1} c_{i}^{j}. 
\end{equation}
gdje je $\hat{X}$ je mjera dobrote klasifikacije, $N$ broj skupova podataka, a $c_{i}^{j}$ izvedba klasifikatora $c^j$ (ovdje $j\in (1,2)$) nad skupom podataka $i$.

\citep{webb2000multiboosting} napominje kako ovakav način često nije smislen, jer usporedba grešaka između različitih skupova podataka iz različitih domena nije nešto što bi trebalo uspoređivati. Ukoliko nije smisleno raditi usporedbe rezultata u različitim domenama, zbranjanje tih vrijednosti također nije valjana operacija. Naravno, postoje slučajevi u kojima je ova metoda ispravna, stoga je ne valja u potpunosti odbaciti.

\subsubsection{Upareni t-test}

Upareni t-test \citep{sprinthall1990basic} uzima u obzir prosječnu razliku u performansama nad različitim skupovima podataka i provjerava ima li statistički značajne razlike. Neka su $c^{1}_{i}$ i $c^{2}_{i}$ rezultati dva klasifikatora na skupu podataka $i$, od ukupno $N$ skupova podataka. Razlika $d_i$ se definira kao $c^{1}_{i} - c^{2}_{i}$. Tada se \textit{t} vrijednost se dobije $\hat{d}/\sigma_{\hat{d}}$, gdje je $\sigma_ {\hat{d}}$ varijanca varijable $d_i$ raspoređenoj prema Studentovoj raspodjeli (engl.\,\textit{Student distribution}) s $N-1$ stupnjeva slobode. 

Tri su temeljne zamjerke korištenja t-testa pri evaluaciji klasifikatora. Prva zamjerka spomenuta je prilikom mjerenja značajnosti računanjem prosjeka. U oba slučaja rade se usporedbe performansi između različitih skupova podataka, ali u ovom slučaju uzima se u obzir varijanca između skupova podataka, što donekle umanjuje standardnu pogrešku $\sigma_d$, no i dalje se radi dvojbena usporedba različitih skupova podataka. U slučaju da je uzorak manji od (približno) 30 jedinki, korištenje t-testa zahtjeva da uzorak zadovoljava uvjete normalne razdiobe. Dakle, problematično je korištenje t-testa sa malim uzorcima, jer nije moguće pouzdano ispitati normalnost nad malim skupom podataka \citep{razali2011power}. Treći problem korištenja t-testa je problem ekstremnih vrijednosti. Ekstremne vrijednosti (jako dobri ili jako loši rezultati) mogu učiniti pomaknuti distribuciju rezultata i tako umanjiti snagu t-testa.

\subsubsection{Wilcoxonov test rangova s predznacima}

Wilcoxonov test rangova s predznacima \citep{wilcoxon1945individual} je neparametarska verzija uparenog t-testa. Razlike u performansama klasifikatora se rangiraju za svaki skup podataka. Rangovi se formiraju prema apsolutnim vrijednostima razlika, a potom se uspoređuju rangovi pozitivnih i negativnih razlika. U slučaju jednakih vrijednosti, u rang se upisuje prosječna vrijednost. Kao što smo opisali kod uparenog t-testa, $d_i$ će predstavljati razliku performansi dva klasifikatora na $i$-tom skupu podataka (od ukupno $N$). Ako razlikujemo klasifikatore $1$ i $2$, potrebno je zbrojiti rangove $R^1$ i $R^2$, prema formulama:

$$ R^1 = \sum_{d_i>0} rank(d_i) + \frac{1}{2} \sum_{d_i=0} rank(d_i) $$

$$ R^2 = \sum_{d_i<0} rank(d_i) + \frac{1}{2} \sum_{d_i=0} rank(d_i) $$.

Nerješeni slučajevi se jednoliko raspoređuju u svaku sumu. Ukoliko postoji neparan broj nerješenih slučajeva, jedan nerješeni slučaj se zanemaruje, kako bi brojka uvijek bila jednaka. Rezultat je statistički značajan ukoliko je je dobivena vrijednost veća od granične $T$ vrijednosti. $T$ vrijednost se dobiva kao $min(R^1, R^2)$ 
Za uzorke manje od 25 moguće je pronaći točnih graničnih $T$ vrijednosti \citep{wilcoxon1973critical}, a za ostale potrebno je izračunati spomenutu $z$ statistiku:
%SPOMENUTI Z statistiku
$$z = \frac{T - \frac{1}{4}N(N+1)}{\sqrt{\frac{1}{24}N(N+1)(2N+1)}}$$

Wilcoxonov test također radi usporedbe različitih skupova podataka, ali ne količinski, zbrajanjem iznosa razlika, već kvalitativno, zanemarujući apsolutne iznose razlika. Pretpostavka da su podatci normalno distribuirani nije potrebna, jer je Wilcoxonov test neparametarski. Utjecaj graničnih vrijednosti je smanjen jer se sada promatraju rangovi, umjesto apsolutnih vrijednosti. 

Wilcoxonov test ima manju snagu od uparenog t-testa kada su podatci normalno distribuirani. U suprotnome, Wilcoxonov test može, ali i ne mora prema \citep{demvsar2006statistical} biti jači od t-testa.

\subsubsection{Test znakova}

Zanemarivanjem vrijednosti razlika u performansama klasifikatora isptivanje značajnosti svodi se na bilježenje rezultata kroz pobjede, poraze i izjednačene rezultate. Ako se uspoređuju dva algoritma na ovaj način, nulta hipoteza pretpostavlja kako će svaki algoritam imati približno $N/2$ pobjeda u skupu od $N$ podataka. Pošto razlikujemo dva ishoda, podatke moguće je pretpostaviti binomnu distribuciju podataka \citep{miller1965probability} i provesti binomni test \citep{dixon1946statistical}, test znakova (engl.\,\textit{sign test}). Postoje izračunate vrijednosti za test znakova vidljive u literaturi \citep{wilcoxon1973critical} prikazane i objašenjene u tablici \ref{tab:crit_val_twotailed_sign}.

\begin{table}
\begin{tabularx}{\textwidth}{c| XXXXXXXXXXXXXXXXXXXXX}
\hline
broj uzoraka & 5&6&7&8&9&10&11&12 & 13&14&15&16&17&18&19&20&21&22&23&24&25 \\
$\alpha = 0.10$ & 0&0&0&1&1&1&2&2&3&3&3&4&4&5&5&5&6&6&7&7&7 \\
\hline
\end{tabularx}
\caption{Kritične vrijednosti za dvostrani test znakova. Ukoliko je broj pobjeda manji ili jednak onome iz drugog retka, razlika je statistički značajna.}
\label{tab:crit_val_twotailed_sign}
\end{table}

U testu znakova ne rade se usporedbe između različitih skupova podataka, dakle nije potrebno zadovoljiti pretpostavku da se podatci mogu/smiju uspoređivati, niti je potrebno da podliježu normalnoj razdiobi. Mana testa znakova jest da neće odbaciti nultu hipotezu ukoliko rezultati jednog algoritma nisu konstatno bolji od drugog. Prema tome, snaga testa znakova je manja od Wilcoxonovog testa.

Prilikom nabrajanja korištenih testova statističke značajnosti u pododjeljku \ref{subsec:couting} još je spomenuto i prebrojavanje statistički značajnih pobjeda/nerješenih rezultata/poraza, što je ekvivalentno ovom načinu, prethodno filtrirajući rezultate nad kojima je utvrđena statistička značajnost. Statistička značajnost ispituje se jednim od testova značajnosti (opisanim u poglavlju %DODAJ REFERENCU NA PRETHODNO POGLAVLJE
). Prema \citep{demvsar2006statistical} ovakav postupak je neispravan, jer pretpostavlja kako je moguće da statistički testovi razlikuju prave od slučajnih razlika. Statistički testovi mjere vjerojatnost dobivenog rezultata pod uvjetom da je nulta hipoteza zadovoljena, što nije ekvivalentno vjerojatnosti nulte hipoteze.

\subsection{Usporedba više klasifikatora na različitom skupu podataka}

Usporedba više od dva klasifikatora podrazumijeva kako više nije moguće provoditi testove zasnovane na uparenom t-testu iz istog razloga zašto nije moguće provoditi uzastopne t-testove -- gomilanje pogreške prvog tipa. \citep{salzberg1997comparing} je primjetio i dokazao kako je moguće primjeniti \textit{Bonferonnijevu metodu} (engl.\,\textit{Bonferroni correction}) i tako omogućiti višestruko testiranje. Bonferronijeva korekcija dozvoljava višestruke usporedbe kontroliranjem razine pogreške. \citep{bland1995multiple}. Primjenom korektivnog faktora Bonferonni korekcija umanjuje $p$ smanjujući mogućnost pogreške tipa I.

Opće statističke metode koje bi se mogle koristiti za usporedbu više klasifikatora na različitim skupovima su analiza varijance (engl.\,\textit{analysis of variance}) ANOVA i Friedmanov test (engl.\,\textit{Friedman test}). 

\subsubsection{Analiza varijance}

%NAGLASITI VIŠESTRUKOST

ANOVA je često korištena metoda za testiranje značajnosti razlika između više od dvije srednje vrijednosti. Opisana je u \citep{fisher1956statistical}. Nulta hipoteza koju ANOVA pretpostavlja je da nema statistički značajne razlike između vrijednosti koje se uspoređuju. Odbacivanje nulte hipoteze nam govori samo da postoji značajna razlika, a ne gdje se nalazi. U tu svrhu osmišljeni su različiti \textit{post-hoc} testovi, primjerice Tukeyev (engl.\,\textit{Tukey's test}) ili Dunnetov test (engl.\,\textit{Dunnett test})\citep{wallenstein1980some}. Tukeyev test uparuje i uspoređuje sve kombinacije klasifikatora, dok Dunnetov test radi usporedbe svih klasifikatora s jednim referentim. Navedeni \textit{post-hoc} testovi u suštini se ne razlikuju od višestruke primjene t-testa,  s time što \textit{post-hoc} testovi kontroliraju mogućnost pogreške tipa I.

Korištenje ANOVA postupka zahtjeva dva uvjeta. ANOVA pretpostavlja kako analizirani uzorci podliježu normalnoj raspodjeli, što često nije slučaj prilikom analize postupaka strojnog učenja. Druga pretpostavka ANOVE je homogenost varijanci (engl.\,\textit{homogeneity of variance}). Svojstvo homogenosti varijanci je ispunjeno ukoliko sve mjere imaju jednake varijance, što se često ne može primjeniti na rezultate klasifikatora. Primjena \textit{post-hoc} testova se ne preporučuje ukoliko pretpostavke ANOVE nisu zadovoljene \citep{zar1974multiple}.

\subsubsection{Friedmanov test}

Kao što je Wilcoxonov test znakova neparametarska verzija uparenog t-testa, tako je Friedmanov test \citep{friedman1937use} neparametarska verzija višestruke ANOVE. U Friedmanovom testu se rangiraju algoritmi za svaki skup podataka zasebno. U slučaju jednakih vrijednosti, dodjeljuju se prosječni rangovi.

Ako je $r_{i}^{j}$ rang $j$-tog klasifikatora na $i$-tom skupu podataka (ukupno $N$). Friedmanov test uspoređuje prosječne rangove klasifikatora:

$$
R_j = \frac{1}{N} \sum_{i}^{N}r_{i}^{j}
$$

Nulta hipoteza u Friedmanovom testu tvrdi da nema razlike između srednjih vrijednosti rangova klasifikatora. Iz ranga $R_j$, stupnjeva slobode $k-1$ Friedmanova vrijednost računa se:

$$
\chi_{F}^{2} = \frac{12N}{k(k+1)}\left[\sum_j R_{j}^{2}-\frac{k(k+1)^2}{4}\right]
$$.

U praksi se češće Friedmanova vrijednost svodi na F-vrijednost koja podliježe F-distribuciji s $k-1$ i $(k-1)(N-1)$ stupnjeva slobode prema formuli:

$$
F_F = \frac{(N-1)\chi_{F}^{2}}{N(k-1)-\chi_{F}^{2}}
$$

Friedman je usporedio svoj test s ANOVA-om kroz eksperiment \citep{friedman1940comparison} i zaključio kako se testovi uglavnom slažu oko rezultata. 

Kada Friedmanov test opovrgne nultu hipotezu, potrebno je nastaviti s \textit{post-hoc testovima}. Nemenyijev test (engl.\,\textit{Nemenyi test}) \citep{nemenyi1963distribution} neparametarska je verzija Tukeyevog testa za ANOVA-u, dakle uspoređuje rezultate svih klasifikatora međusobno. Performanse dvaju klasfikatora se značajno razlikuju ukoliko se prosjeci njihovih rangova razlikuju za barem:

$$
CD = q_{\alpha}\sqrt{\frac{k(k+1)}{6N}}
$$

U slučaju da je potrebno uspoređivati sve klasifikatore s jednim referentnim, moguće je koristiti proceduru koja primjenom Bonferonnijevog korektivnog faktora radi međusobne usporedbe. U ovom slučaju radi se manje usporedbi ($k-1$), naspram $k(k-1)/2$ prilikom izvedbe Nemenyjevog testa. Što se radi manje usporedbi, to je test jači, zato se i prilikom usporedbi s referentnim klasifikatorom preferira ovakav način usporedbe. Usporedba $i$-tog i $j$-tog klasifikatora se računa:
$$
z = \frac{R_i - R_j}{\sqrt{\frac{k(k+1)}{6N}}}.
$$
Dobivena $z$ vrijednost se uspoređuje s odgovarajućim vrijednostima normalne distribucije za utvrđivanje značajnosti. 

U radu \citep{demvsar2006statistical} analizirani su Nemenyijev test s testom koji koristi Bonferonni korekciju. Zaključeno je kako je snaga testa veća ukoliko se uspoređuju svi klasifikatori s jednim referentnim. Međusobna usporedba svih klasifikatora se jedino preporučuje u slučaju kada je potrebno dokazati da su novootkrivene tehnike bolje od svih postojećih.

\section{Metode procjene p-vrijednosti}

U dosadašnjem dijelu odgovaranja na statistička pitanja iz strojnog učenja uvijek je bilo moguće analitički izračunati relevantnu statističku mjeru ili željenu $p$-vrijednost. U slučajevima kada nije moguće ili nije ispravno zbog razloga opisanih u odjeljku \ref{sec:critique} kao alternativna opcija nameće se procjena $p$-vrijednosti. U narednom pododjeljku opisat će se takva tehnika nazvana \textit{bootstrap}.

\subsection{Bootstrap}

Bootstrap tehnikom moguće je procijeniti \textit{p}-vrijednost \citep{thompson1993use}. Prema Thompsonu, ispravnije je koristiti bootstrap metodu od analtičkih metoda za dokazivanje statističke značajnosti. Ovdje ćemo opisati primjenu boostrap metode u strojnom učenju. To je samo jedan način provedbe boostrap metode, generalno opisane u \citep{efron1993introduction}. 

U boostrap metodi \citep{berg2012empirical} uzimaju se uzorci $x_i$ iz populacije i mjere  $f_{A}^{i}$ i $f_{B}^{i}$: performanse klasifikatora $C_A$ i $C_B$. $\delta(x_i) = f_{A}^{i} - f_{B}^{i}$ je očekivana razlika u performansama između klasifikatora $A$ i $B$. Zbog ograničenog skupa podataka podaci iz $S$ uzorkuju se sa zamjenom \engl{sampling with replacement}. Tako dobiveni uzorci nazivaju se \textit{bootstrap} uzorcima.

Za dobivene uzorke $x_i$, prema nultoj hipotezi, bi trebala vrijediti jednakost:
\begin{equation}
\label{eq:jednakost_delta}
\frac{1}{k}\sum_{i=0}^{k}\delta(x_i) = \delta(x)
\end{equation}

gdje je \textit{k} broj uzoraka. Ako se želi provjeriti može li se odbaciti nulta hipoteza, potrebno je provjeriti \textit{koliko često klasifikator $C_A$ daje rezultate koji su bolji od očekivanih}. Očekivani rezultat je da je klasifikator $C_A$ bolji od klasifikatora $C_B$ za $\delta(x)$. Prema tome, potrebno je prebrojati u koliko je slučajeva \textit{A} bio bolji od $C_B$ za $2\delta(x)$. Pseudokod bootstrap postupka je prikazan je \citep{berg2012empirical} bootstrap postupka prikazan je \ref{code:bootstrapkod}. U završnom koraku dolazi se do procjenjene $p$-vrijednosti na temelju koje se donosi odluka o odbacivanju nulte hipoteze. Najveća prednost bootstrap metode je mogućnost računanja $\delta(x)$ za bilo koju metriku. 

\begin{algorithm}
\caption{Pseudokod bootstrap postupka}\label{code:bootstrapkod}
\begin{algorithmic}[1]
\State Generiranje $M$ bootstrap uzoraka $x_i$ veličine $n$ nasumičnim izborom s ponavljanjem iz populacije $S$
\State $s=0$
\Repeat 
\If{$\delta(x_i)>2\delta(x)$}
\State $s=s+1$
\EndIf
\Until{$i>n$}
\State $p=\frac{s}{M}$
\end{algorithmic}
\end{algorithm}



