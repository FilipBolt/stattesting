Bruce Thompson \citep{thompson1993use} navodi neke kritike konvencionalnih metoda statističkog testiranja:

1. Nulta hipoteza će se \textbf{uvijek} odbaciti, ako uzme u obzir dovoljno velika populacija. Thompson kaže:
	
\begin{quote}
Ispitivanje statističke značajnosti može biti vođeno tautologijom. Umorni istraživači, nakon prikupljanja skupa podataka, rade statističke testove kako bi se uvjerili da je prikupljena dovoljno velika količina podataka, što je podatak koji već znaju. Ovakva tautologija učinila je mnogo štete znanstvenoj zajednici.
\end{quote}

2. Korištenje ANOVA-e može dovesti do pogrešnih usporedbi. U višedimenzionalnim analizama primjenom hijerarhijskog pristupa moguće je povlačiti usporedbe između podataka iz različitih dimenzija, analogno poslovici: usporedba krušaka i jabuka. 

3. Oslanjanje na ispitivanje statističke značajnosti stvara neizbježne dvojbe. \\  ANOVA zahtjeva spajanje varijanci prilikom izračuna srednje devijacije (u nazivniku). Ova operacija je dozvoljena samo u slučaju da su varijable homogene, međusobno usporedive. Slično tome, ANCOVA (analiza kovarijance) pretpostavlja da je zadovoljen uvjet \textit{homogenosti regresije} \engl{Homogeneity of Regression Slopes}. 



%TODO napisati za ostale metode osim uparenog bootstrapa

\subsection{Bootstrap}

Bootstrap procjenjuje \textit{p-vrijednost}, vadi testne uzorke $x_i$ iz populacije i broji koliko često sustav \emph{A} funkcionira s performansama $\delta(x)$ (ili većim) od sustava \textit{B}. Na raspolaganju stoji samo populacija $x$, pa se podaci iz $x$ uzorkuju sa zamjenom \engl{sampling with replacement}. Tako dobiveni uzorci nazivaju se \engl{bootstrap} uzorcima. 

Za dobivene uzorke $x_i$ bi, prema nultoj hipotezi, trebalo vrijediti da 
\begin{equation}
\label{eq:jednakost_delta}
\frac{1}{k}\sum_{i=0}^{k}\delta(x_i) = \delta(x)
\end{equation}

gdje je \textit{k} broj uzoraka. Ako se želi provjeriti može li se odbaciti nulta hipoteza, potrebno je provjeriti \textit{koliko često sustav A daje rezultate koji su bolji od očekivanih}. Očekivani rezultat je da je sustav \textit{A} bolji od sustava \textit{B} za $\delta(x)$. Prema tome, prebrojava se u koliko slučajeva test skupova podataka $x_i$ je \textit{A} bio bolji od \textit{B} za $2\delta(x)$. Pseudokod bootstrap postupka je prikazan je \citep{berg2012empirical} bootstrap postupka prikazan je \ref{code:bootstrapkod}.

%\begin{algorithm}
%\caption{Pseudokod bootstrap postupka}\label{code:bootstrapkod}
%\begin{algorithmic}[1]
%\State Generiranje $b$ bootstrap uzoraka $x_i$ veličine $n$ nasumičnim izborom s ponavljanjem iz populacije $x$
%\State $s=0$
%\Repeat 
%\If{$\delta(x_i)>2\delta(x)$}
%\State $s=s+1$
%\EndIf
%\Until{$i>n$}
%\State $p=\frac{s}{b}$
%\end{algorithmic}
%\end{algorithm}
%
Najveća prednost bootstrap metode je mogućnost računanja $\delta(x)$ za bilo koju metriku. 

