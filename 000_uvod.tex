U obradi prirodnog jezika \engl{natural language processing - NLP} verifikacija izgrađenog sustava (za prevođenje, klasifikaciju govornih činova, izgradnje stabla međuovisnosti \dots) nužan je korak. Uz argumentirano obrazloženje autora sustava o dobivenim performansama sustava, potrebno je priložiti statistički dokaz kojime se nepobitno pokazuje kako je uistinu ostvaren doprinos znanstvenoj zajednici. Ispitivivanjem statističke značajnosti pokazuje se koliko su rezultati eksperimenata vrijedni, jesu li dobiveni slučajno i u kolikoj mjeri su pouzdani. Ovaj korak je sastavni dio velikog broja NLP članaka.

\citep{chinchor1992statistical} se bavi analizom rezultata MUC-4 \engl{Message Understanding Conference}, \citep{koehn2004statistical} i \citep{zhang2004interpreting} ispituju značajnost rezultata strojnog prevođenja \engl{machine translation}, \citep{bisani2004bootstrap} analiziraju rezultate automatskog prepoznavanja govora \engl{automated speech recognition}. \citep{berg2012empirical}, \citep{yeh2000more} , \citep{thompson1993use} se nisu usredotočili na specifičnu metriku (kao što je F1-mjera), već općenitijim metodama ispitivanja statističke pouzdanosti.

U nastavku seminarskog rada opisat će se različiti načini ispitivanja statističke značajnosti iz navedenih radova.  

mos mislit