Dokazivanje statističke značajnosti neizostavan je dio svake znanstvene publikacije, pa tako i publikacija u području obrade prirodnog jezika \engl{natural language processing -- NLP}. Neki od problema u sklopu obrada prirodnog jezika su klasifikacija govornih činova \engl{speech acts} \citep{pratt1977toward}, izgradnja stabla međuovisnosti \engl{dependency tree} \citep{collins2003head}, sažimanje teksta \engl{text summarizing}, ekstrakcija ključnih riječi \engl{keyword extraction} i drugi. Dobivene rezultate za, primjerice, ekstrakciju ključnih riječi potrebno je verificirati. Nije dovoljno riječima argumentirati u korist rezultata već, pošto je postala ustaljena praksa u znanstvenoj zajednici, je potrebno provesti verifikaciju testiranjem statističkih hipoteza. Korištenje statističkih testova u istraživanju standardizira provođenje eksperimenata i daje dodatnu informaciju o rezultatima eksperimenta.

Dobro provedeno istraživanje mora biti potkrijepljeno kvalitetnim statističkim analizama rezultata. Kvalitetna analiza podrazumijeva da pružanje dokaza o tome da su eksperimenti predstavljeni u radu dobiveni nad reprezentativnim podatcima, da nije bilo utjecaja 

U obradi prirodnog jezika  verifikacija izgrađenog sustava (za prevođenje, klasifikaciju govornih činova, izgradnje stabla međuovisnosti \dots) nužan je korak. Uz argumentirano obrazloženje autora sustava o dobivenim performansama sustava, potrebno je priložiti statistički dokaz kojime se nepobitno pokazuje kako je uistinu ostvaren doprinos znanstvenoj zajednici. Ispitivivanjem statističke značajnosti pokazuje se koliko su rezultati eksperimenata vrijedni, jesu li dobiveni slučajno i u kolikoj mjeri su pouzdani. Ovaj korak je sastavni dio velikog broja NLP članaka.

\citep{chinchor1992statistical} se bavi analizom rezultata MUC-4 \engl{Message Understanding Conference}, \citep{koehn2004statistical} i \citep{zhang2004interpreting} ispituju značajnost rezultata strojnog prevođenja \engl{machine translation}, \citep{bisani2004bootstrap} analiziraju rezultate automatskog prepoznavanja govora \engl{automated speech recognition}. \citep{berg2012empirical}, \citep{yeh2000more} , \citep{thompson1993use} se nisu usredotočili na specifičnu metriku (kao što je F1-mjera), već općenitijim metodama ispitivanja statističke pouzdanosti.

U ovom radu prvo će se objasniti osnove ispitivanja statističke značajnosti. Nakon toga pričat će se o mogućim testovima značajnosti u području strojnog učenja. Svaki spomenuti test bit će kratko objašnjen, s pružanjem prikladnih referenci. Objasnit će se mehanizam pojedinih statističkih testova te prikladnost ili neprikladnost statističkog testa u kontekstu specifičnog problema. Spomenuti problemi bit će tipični problemi iz područja strojnog učenja, s blagim naglaskom na obradu prirodnog jezika. Konačan cilj rada je dati čitatelju uvid u aktualne statističke testove nad problemima strojnog učenja, kako bi što bolje odabrao metodologiju ispitivanja statističku značajnost vlastitih eksperimenata.