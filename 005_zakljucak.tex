U seminarskom radu pružen je pregled metoda ispitivanja statističke značajnosti u nekim problemima iz bogatog područja strojnog učenja. Rad ima svrhu pružanja referentnog mjesta za provođenje statističkih provjera istraživačima u području strojnog učenja, ali i demonstrira koliko izbor testa može biti rezultirati različitim rezultatom ispitivanja u istim okolnostima eksperimenta. Prepoznavanje uvjeta eksperimenata je zadaća istraživača, što je preduvjet za ispravan odabir odgovarajućeg statističkog testa. 

Daljnje istraživanje vezano uz statističko ispitivanje značajnosti moglo bi dovesti do usuglašavanja standarda za provođenje. Izgradnjom standarda, otvara se i put automatizaciji samog procesa statističkog testiranja. Krajnji proizvod takve automatizacije mogao bi biti softver koji automatski prepoznaje uvjete eksperimenta te provodi ispitivanje statističke značajnosti nad ulaznim podatcima (primjerice, rezultatima algoritma učenja). Takav softver vjerojatno bi trebao biti zasnovan na modelima strojnog učenja upravo iz razloga što nije moguće popisati i predvidjeti sve moguće uvjete viđene u eksperimentima, stoga bi automatski proces morao imati ugrađene mehanizme na temelju kojih može kategorizirati neviđene kombinacije uvjeta eksperimenta, u čemu se strojno učenje pokazalo vrlo dobrim. Ovo je jedan mogući cilj daljnjeg istraživanja statističke značajnosti.

Ispravno korištenje metoda ispitivanja statističke značajnosti pomaže znanstvenoj zajednici pri ispravnom tumačenju statističkih podataka. Zablude u znanstvenim publikacija nerijetko se objave u znanstvenim krugovima, pa čak i dospiju do senzacionalističkih naslova u medijima. Do takvih pogrešaka uglavnom dolazi zbog nedovoljno pažljive validacije rezultata. Bolja, stroža i standardizirana kontrola rezultata eksperimenata trebala bi dovesti do povećane kvalitete znanstvenih radova te uspješnijem i bržem znanstvenom napretku.